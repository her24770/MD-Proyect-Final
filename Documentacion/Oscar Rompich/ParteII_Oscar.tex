\documentclass[12pt]{article}\usepackage[spanish]{babel}
\usepackage[utf8]{inputenc}
\usepackage{graphicx}
\usepackage{amsmath}
\usepackage{amssymb}
\usepackage{geometry}
\usepackage{hyperref}
\usepackage{float}
\usepackage{caption}
\usepackage{enumitem}
\usepackage{fancyhdr}
\usepackage{appendix}

\geometry{margin=1in}
\pagestyle{fancy}
\fancyhf{}

\rfoot{Página \thepage}

\title{Aplicaciones de la Matemática Discreta en el Contexto de Ciencias de la Computación}
\author{Oscar Rodrigo Rompich Cotzojay\\}
\date{Guatemala, 9 de noviembre 2025}

\begin{document}

% Carátula
\begin{titlepage}
    \centering
    \vspace*{2cm}
    
    {\huge\bfseries Aplicaciones de la Matemática Discreta en el Contexto de Ciencias de la Computación\par}
    
    \vspace{2cm}
    
    {\Large\itshape Oscar Rodrigo Rompich Cotzojay\par}
    
    \vspace{1.5cm}
    
    {\large Univesidad del Valle de Guatemala\par}

    {\large Matemática Discreta\par}
    
    \vfill
    
    {\large Guatemala, 9 de noviembre 2025\par}
\end{titlepage}

\newpage

\section{Aplicaciones de la Matemática Discreta en Ciencias de la Computación}

La Matemática Discreta es muy importante en las Ciencias de la Computación, proporcionando una gran 
cantidad de herramientas teóricas para resolver problemas computacionales. A continuación hay algunas 
aplicaciones interesantes investigadas con apoyo de inteligencia artificial generativa para tener un 
punto de partida claro.

\subsection{Teoría de Grafos}

Los grafos modelan relaciones entre entidades. En redes sociales, los usuarios son vértices 
y las conexiones son aristas. Algoritmos como BFS y DFS permiten encontrar caminos óptimos y detectar comunidades.
En redes computacionales, algoritmos como Dijkstra optimizan el enrutamiento de paquetes.

\subsection{Lógica Proposicional}

La lógica booleana fundamenta el diseño de circuitos digitales. Las operaciones AND, OR, NOT, XOR se combinan para c
rear circuitos complejos. Un ejemplo es el sumador completo:
$$S = A \oplus B \oplus C_{in}$$

\subsection{Teoría de Conjuntos y Bases de Datos}

El álgebra relacional se basa en operaciones de conjuntos: unión ($\cup$), intersección ($\cap$), diferencia ($-$), y 
producto cartesiano ($\times$). Estas operaciones permiten consultas eficientes en bases de datos relacionales.

\subsection{Combinatoria}

Analiza la complejidad de algoritmos mediante conteo. Por ejemplo, el ordenamiento de burbuja realiza $\frac{n(n-1)}{2}$ 
comparaciones. Los árboles binarios tienen $C_n = \frac{1}{n+1}\binom{2n}{n}$ configuraciones posibles.

\section{Conclusiones}

\begin{enumerate}
    \item La Matemática Discreta proporciona fundamentos teóricos esenciales para las Ciencias de la Computación.
    \item La teoría de grafos modela eficientemente problemas de redes y optimización.
    \item La lógica proposicional es la base del hardware computacional moderno.
    \item Las operaciones de conjuntos fundamentan las bases de datos relacionales.
    \item El uso de herramientas de IA generativa facilitó la investigación, pero requirió validación crítica de la información obtenida.
\end{enumerate}

\begin{thebibliography}{9}

\bibitem{rosen2019}
Rosen, K. H. (2019). 
\textit{Discrete Mathematics and Its Applications} (8th ed.). 
McGraw-Hill Education.

\bibitem{cormen2009}
Cormen, T. H., Leiserson, C. E., Rivest, R. L., \& Stein, C. (2009). 
\textit{Introduction to Algorithms} (3rd ed.). 
MIT Press.

\end{thebibliography}

\appendix
\appendixpage
\addappheadtotoc

\section{Prompts y Respuestas de IA Generativa}

En esta sección se documentan todas las interacciones realizadas con herramientas de inteligencia artificial generativa durante el proceso de investigación, incluyendo los prompts utilizados y las respuestas obtenidas.

\subsection{Herramientas Utilizadas}
\begin{itemize}
    \item ChatGPT (OpenAI)
\end{itemize}

\subsection{Interacciones Documentadas}

\subsubsection{Interacción 1: Consulta inicial sobre puntos a abordar}

\textbf{Herramienta:} ChatGPT

\textbf{Prompt:} ``Quiero empezar una pequeña investigación sobre el tema de las aplicaciones de la Matemática Discreta en el contexto de ciencias de la computación. ¿Cuáles son los puntos más importantes que debo abordar, desde dónde debería partir y qué fuentes me recomiendas consultar?''

\textbf{Captura de pantalla de la respuesta:}

\begin{figure}[H]
    \centering
    \includegraphics[width=0.85\textwidth]{PuntosfundamentalesAbordar.png}
    \caption{Respuesta de ChatGPT sobre puntos fundamentales que debes abordar}
\end{figure}

\subsubsection{Interacción 2: Recomendación de fuentes de partida}

\textbf{Captura de pantalla:}

\begin{figure}[H]
    \centering
    \includegraphics[width=0.85\textwidth]{DesdeDondePartir.png}
    \caption{Recomendaciones de ChatGPT sobre desde dónde comenzar la investigación}
\end{figure}

\subsubsection{Interacción 3: Fuentes complementarias en línea}

\textbf{Captura de pantalla:}

\begin{figure}[H]
    \centering
    \includegraphics[width=0.85\textwidth]{FuentesEnLinea.png}
    \caption{Fuentes complementarias en línea sugeridas por ChatGPT}
\end{figure}

\subsection{Análisis de las Respuestas Obtenidas}

Las respuestas de ChatGPT proporcionaron:
\begin{itemize}
    \item Una estructura clara para tomar como punto de partida.
    \item Recomendaciones de literatura académica reconocida.
    \item Orientación sobre áreas específicas de aplicación en ciencias de la computación
\end{itemize}

\section{Reflexión Metacognitiva}

\subsection{¿Cómo evaluaste la credibilidad, relevancia y solidez lógica de las respuestas?}

\textbf{Ejemplo de afirmación fiable:}

``El algoritmo de Dijkstra tiene complejidad $O((V + E)\log V)$ con cola de prioridad.''

Esta afirmación fue fiable porque:
\begin{itemize}
    \item Es verificable en literatura académica que anteriormente mencionó.
    \item Consistente entre diferentes IAs
    \item Es Matemáticamente demostrable
\end{itemize}

\textbf{Ejemplo de afirmación que generó dudas:}

``Todos los problemas de grafos se resuelven eficientemente con algoritmos greedy.''

Generó dudas porque:
\begin{itemize}
    \item Es una generalización excesiva
    \item Al pedir aclaración, la IA corrigió su respuesta
\end{itemize}

\subsection{¿La primera respuesta fue suficiente? ¿Cómo refinaste tus preguntas?}

No, las primeras respuestas fueron demasiado generales. Apliqué refinamiento iterativo:
\begin{itemize}
    \item Pregunté por ejemplos específicos en cada área (e.g., grafos, lógica).
    \item Solicité referencias académicas concretas.
    \item Pedí aclaraciones sobre términos técnicos.
\end{itemize}


\subsection{¿Identificaste errores, sesgos u omisiones? ¿Cómo reaccionaste?}

Sí, noté omisiones en explicaciones técnicas. Reaccioné:
\begin{itemize}
    \item Verificando con fuentes académicas.
    \item Solicitando explicaciones más detalladas.
    \item Contrastando respuestas entre diferentes herramientas de IA.
\end{itemize}

\end{document}